\begin{abstract}
Eye-tracking devices can tell us where on the screen a person is looking. Researchers frequently analyze eye-tracking data manually, by examining every frame of a visual stimulus used in an eye-tracking experiment so as to match 2D screen-coordinates provided by the eye-tracker to related objects and content within the stimulus. Such task requires significant manual effort and is not feasible for analyzing data collected from many users, long experimental sessions, and heavily interactive and dynamic visual stimuli. In this dissertation, we present a novel analysis method. We would instrument visualizations that have open source code, and leverage real-time information about the layout of the rendered visual content, to automatically relate gaze-samples to visual objects drawn on the screen. Since such visual objects are shown in a visualization stand for data, the method would allow us to necessarily detect data that users focus on or Data of Interest (DOI). 

This dissertation has two contributions. First, we demonstrated the feasibility of collecting DOI data for real life visualization in a reliable way which is not self-evident. Second, we formalized the process of collecting and interpreting DOI data and test whether the automated DOI detection can lead to research workflows, and insights not possible with traditional, manual approaches.
\end{abstract}