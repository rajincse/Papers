\section{Introduction}
Due to the differences between DOI and AOI, methods for visualizing and analyzing AOI data are unsuited for the analysis of DOI data. The differences are due to two factors. First, DOIs can be more granular and DOI data much larger compared to AOIs. For example, an eye-tracking visualization instrumented with DOI can track 100 DOIs per frame. Conversely, experimenters can annotate only about 10-20 AOIs manually. Thus, AOI-based visualizations are not able to show the typically large volumes of DOI data. For example, Figure~\ref{fig:Scanpath} illustrates a scanpath, a typical AOI visualization,  build from a full DOI dataset including 108 tracked DOIs (Figure~\ref{fig:Scanpath}(a)). Moreover, one build from only twelve DOIs (Figure~\ref{fig:Scanpath}(b)), a count more typical of traditional AOI analyses. We generated both diagrams from an eye-tracking session of ten minutes. It is evident that the scanpath diagram for the full DOI set is more complicated than that representative of AOI analyses. Thus, interpreting large DOI data from such diagrams is difficult. 

\begin{figure}[htb]
  \centering
  \includegraphics[width=0.99\linewidth]{images/Scanpath.eps}
  \caption{Two examples of scanpath visualization. Here, (a) displays scanpath with 108 DOIs. (b) displays scanpath with 12 AOIs. }
	\label{fig:Scanpath}
\end{figure}

\begin{table}[htbp]
\caption{Three attributes (isAlive,Gender, and Centrality) for each character in the Les Miserables Data. }
	\centering
		\begin{tabular}{|c|c|c|c|}
				\hline
				\textbf{Character}	& \textbf{Survives} &	\textbf{Gender}	& \textbf{Centrality}\\\hline
			
				Valjean	& No	&Male	&1\\\hline
Javert	&No&	Male&	3\\\hline
Cosette 	&Yes	&Female&	2\\\hline
Marius	&Yes	&Male&	6\\\hline
Eponine	&No	&Female&	4\\\hline
Fantine	&No	&Female	&5\\\hline
Thanerdier	&Yes	&Male	&7\\\hline

		\end{tabular}
		
		\label{tab:LesMiserablesAttribute}
\end{table}

Second, since DOIs are data-derived, they have specific attributes. A set of attributes that the visualization can access, display, and leverage describes each DOI. Instead, AOIs have implicit attributes that only experimenters that defined them know. Such attributes are inaccessible to visualizations and analysis software. For example, consider that in `The Les Miserables Visualization' example (Figure~\ref{fig:MiserablesSimple}), attributes (Table~\ref{tab:LesMiserablesAttribute}) such as `Centrality', `Gender', and `Survives' (i.e. whether the character is alive at the end of the novel) describe each character. Once they collect data about which objects subjects viewed, analyzers can answer a breadth of questions such as ``Are users more likely to view female characters rather than male characters?'', ``Are users more likely to view deceased characters rather than alive?'', alternatively, ``Do users tend to look at central characters?''. 

In conclusion, DOI data interpretation is different from AOI data. Hence, we aim to formalize DOI data model and analysis tasks. In this chapter, we discuss our contributions to a general data model in Section~\ref{sec:DOIDataModel} and a list of analysis questions for DOI data in Section~\ref{sec:DOIAnalysisTasks}. Moreover, we discuss a comparison between AOI and DOI data in Section~\ref{sec:AOIvsDOI}. Finally, we conclude our discussion about this chapter in Section~\ref{sec:DOIAnalysisModelConclusion}.


