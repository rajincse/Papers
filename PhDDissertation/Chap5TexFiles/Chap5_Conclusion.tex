\section{Conclusion}
\label{sec:DOIAnalysisModelConclusion}
DOIs are subsets of data that a visual interface shows to a user. We define them on the data model underlying a visualization, by instrumenting the code that translates concrete data into visual representations. Once a visual interface instrumented, user gaze coordinates provided by an eye-tracker can be mapped to DOIs via their visual representations automatically and effortlessly, regardless of users' individual interactions with the interface. As such, the DOI approach can capture users' data interests from interactive visualizations over extended periods of time. Moreover, DOIs are characterized by a rich set of attributes derived from the data that the DOIs are defined on, and from the visual context in which they are displayed. These attributes allow analysts to pose a broad range of questions that relate the type of viewed data to user behavior and characteristics. While DOIs can be regarded as a mere extension of AOIs, there are significant differences in DOI data properties and how it is collected, the research goals it can support, and the data questions it facilitates. Moreover, current visualization techniques do not help DOI-specific analyses. These differences, justify the different nomenclature and motivate the research.    

