\section{Introduction}
The motivation of using DOI analyses over traditional methods (i.e. point-based and AOI-based) is that it will significantly reduce human interventions for analysis processes. We claim that using DOI analyses over traditional AOI analysis will have four major advantages. First, experimenters will be able to analyze a user study with longer sessions. A traditional eye-tracker reports data in 60Hz to 120Hz. Hence, for an hour-long session, experimenters may have to examine $ 60 \times 1 \times 60 \times 60 =216,000$ gaze points for a single user. Although, automated interpretation visualization tools exist with contemporary eye-tracker software packages. However, such analysis tools can only create visualizations for the entire experimental sessions and can identify only screen locations. For example, in an eye-tracking experiment, a user was looking at a diagram for one hour would have produced gaze points recorded all over the given stimulus. It cannot identify which visual elements the user was viewing. For that, experimenters would have to go over every gaze points over time manually. DOI analyses data would contain time annotated data elements. Hence, it will be possible to analyze eye-tracking data for longer sessions.  

Second, DOI analyses can handle eye-tracking data from more users than traditional methods can handle. DOI analyses eliminate the process of manually relating gaze points with semantic contents of given stimuli.  For example, data interpretation of a user of an eye-tracking study may take 5-6 hours. Thus, a user study with ten subjects will require analysis for 50-60 hours. DOI analyses automatically relate gaze points with semantic contents which eliminate such exhaustive process. Hence, DOI analyses enable experimenters to conduct user studies with more subjects than it was possible.

Third, DOI analyses enable experimenters to use complex, interactive, and dynamic visualizations for eye-tracking user studies. We discussed that experimenters have to spend a significant amount to time to relate gaze points with given stimulus. However, if a stimulus is interactive and dynamic, then experimenters have to repeat the same task for each frame. Moreover, the process is even more difficult for dense and complicated visualization layout. DOI analyses facilitate experimenters by providing data elements that users were interested in real-time.

Fourth, experimenters can test users with more open-ended tasks in eye-tracking studies with DOI-based analysis. Traditional eye-tracking analysis methods can provide a relatively small amount of analysis data compared to DOI data. Thus, they compel experimenters tends to use small close ended tasks. However, DOI analyses can provide detail reports of elements that users tend to see. Thus, more behavioral analyses are possible with DOI-based analysis. 

Again, DOI data collection is the process of relating gaze points to data elements. With the open-source code for generating visualizations, we know layouts of all visual elements and their corresponding data elements. However, eye-trackers report data with low-resolution and inaccuracy. Due to peripheral vision, a human can view an area rather than a precise pixel. Thus, identifying objects a user viewed is challenging. We discuss the difficulty of DOI data collection in Section~\ref{sec:DOINonTrivial}. 

We implemented a method of fuzzy interpretation of gaze data. The method reports a likelihood of viewing an object rather than certainty. Using this approach, we implemented a novel viewed-object-detection algorithm. We discuss the incremental development of this algorithm and instrumentation to visualization code process in Section~\ref{sec:DOICollectionMethods}. 

To test our algorithm, we have conducted an eye-tracking experiment with instrumented DOI data collection code. Details of the experimental setup and results are discussed in Section~\ref{sec:DOICollectionEvaluation}. Finally, we provide our conclusion remarks in Section~\ref{sec:DOICollectionConclusion}.