\chapter{CONCLUSIONS}
\label{chap:Conclusion}
\section{Summary}
Eye-tracking data analysis is challenging and exhaustive. In this dissertation, we proposed a new analysis method of eye-tracking data: DOI-based analysis. DOIs are data space counterpart for AOIs. However, DOI data tend to be larger, more granular, and richly annotated than AOI data. 

The DOI-based analysis is limited within the visualization with open source code which can be instrumented. In such visualization, we define DOIs in the underlying data model of it. In an eye-tracking experiment with DOI instrumented visualization, experimenters can collect automated analysis data which is not possible with traditional methods. Moreover, DOI data can answer analysis question that is not feasible for eye-tracking data collected by conventional methods. 

We also demonstrated that collecting accurate DOI data is feasible. We evaluated our claim by collecting DOI data from three different eye-tracking experiments. Each experiment used highly interactive and dynamic visualizations. We also described detail instrumentation methods and study design for each study.

Next, we formalized DOI-based analysis. First, we described a data model for DOI. We defined DOI in an entity and value style consisting data model, eye-tracking gaze points, and user data. Second, we listed possible and probable questions that DOI data can answer. We categorized the questions and provided examples based on the three experiments we conducted. Third, we discussed visual design guidelines for interpreting DOI data. We discussed existing visual techniques and proposed possible modifications over them for DOI data. We also demonstrated several visual technique instances for interpreting DOI data from the three experiments. Moreover, we discussed interaction techniques on each of them.

DOI data can enable data-driven, exploratory eye-tracking research not previously possible, by supporting long ``in vivo'' experiments of complex and interactive visual content. The dissertation creates a foundation for such research by formalizing DOI data and tasks and provides visual design guidelines in supporting DOI analyses.

\section{Future Work}
The research described in this dissertation can be improved in three possible directions. First, the viewed-object-detection algorithm described in Chapter~\ref{chap:DOIDataCollection} assume every visual element as rectangles. We generate every calculation based on the position of a visual element and its dimension. However, in real-life, visual elements can come in different shapes and dimensions. Moreover, our algorithm does not consider the case of human visual perception changes to various sizes of visual objects. We can improve our viewed-object-detection algorithm to support human perception pattern on any shapes and sizes. 

Second, we can build a library for experimenters to enable easy instrumentation. Moreover, the library could support widgets for data interpretation once it is collected. Analyzers will be able to test their hypotheses by using its built-in visualization techniques. Third, using faster DOI data collection and cheap eye-trackers, we can experiment with real-time eye-tracking applications. We did a pilot study on monitoring an eye-tracking user data instance which itself was tracked by an eye-tracker~\cite{alam2015they}. Such experimenters can be valuable to education research. 