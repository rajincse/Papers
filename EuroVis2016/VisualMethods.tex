\section{A visual design space for DOI analysis}
The previous sections highlighted the differences between AOI and DOI data, and illustrated the broad range of DOI enabled tasks. This provides an intuition that methods previously developed for AOI analysis do not directly extend to DOI data. First, AOI visualizations such as scanpaths, scarfplots, and AOI-rivers cannot handle the scale of DOI data, which is captured at high granularities and over extended periods of time. Second, AOI visualizations were not designed to show or leverage the many attributes that can typically characterize DOIs.  Third, collecting users' interest in richly annotated DOIs over hour long analysis and from many users can support analytic workflows different from those typical of AOI analyses. This section explores a design space of visual solutions that can alleviate these shortcomings.

Broadly, novel visualizations need to allow analysts to explore DOI data at multiple levels of detail, including a range of temporal scales (e.g., several seconds vs. several minutes) and data granularities (e.g., individual DOIs vs. categories of DOIs).  Moreover, attributes need to be shown visually and queried flexibly to allow analyzers to identify links between users' interest in data and that data's properties. DOIs attributes should be leveraged to help deal with the scale of the data by allowing users to filter, highlight, and aggregate data with specific attribute configurations. Finally, visual analyses need to support the particular tasks that DOIs support (section~\ref{sec:Taxonomy}). 

In the following sections we analyze the design space for visual DOI analysis in more detail. Given the scale and complexity of DOI data and the breadth of tasks, visual encodings need to be augmented with significant interaction capabilities. To ensure that our discussion is exhaustive and captures all visual design dimensions, we ground it in Yi's taxonomy of interaction tasks~\cite{yi2007toward}, which includes seven categories of visual design coordinates: Encoding, Selection, Reconfiguration, Exploration, Abstraction/Elaboration, Filtering, and Connection/Comparison~\cite{yi2007toward}. Moreover, to leverage a familiar frame of reference, we will discuss novel visual solutions by starting from existing AOI visualizations and describing how they may be extended and redesigned to support DOI requirements. 

\subsection{Encoding}
\label{sec:Encoding}
Encoding determines how data should be represented visually~\cite{yi2007toward}. Encodings of DOI data need to scale to many DOIs viewed over long periods of time, be able to show data at multiple scales of time and data granularity, show DOI attributes and leverage them to improve legibility, and support the tasks exemplified in section~\ref{sec:Taxonomy}. Encodings should as much as possible support tasks visually, rather than rely extensively on interactions such as navigation or information on demand.

\textbf{Showing many DOIs :} First, some AOI visualizations such as scarfplots and AOI rivers identify individual AOIs through distinctive colors. This method does not scale to hundreds of granular DOIs that users can view during long experiments. Methods that identify DOIs explicitly (e.g., through a label), such as scanpaths or transition matrices, are thus preferable. 

Second, showing all DOIs viewed during an experiment is not always possible, especially when analyzing data from multiple users. For example, a scan-path of ten users, each having viewed approximately one hundred DOIs would need to show a thousand rows and would be extremely cluttered. A solution is to show only the top data and scale data based on how much it was viewed. This is likely to give good results as Alam et al. has shown that even though users view many things, they tend to focus predominantly on a few that are relevant to their momentary goals or analyses. This prioritization can be combined with an ability to specify a time-window of interest, as shown in Figure X, and described in more detail in section X. 

Third, DOI attributes make it possible to aggregate multiple DOIs into categories to reduce the amount of information shown. Such aggregations can be done semantically, by allowing analysts to explicitly collapse multiple DOIs into single ones, such as in hierarchical visualizations for instance~\cite{kurzhals2014iseecube}, or visually, by showing all data in a way that allows categories to emerge and separate, such as for example in pixel based techniques~\cite{keim2000designing}.

\textbf{Showing long experiments :} Showing time at multiple resolutions and restricting views to time windows are addressed in sections X and Y and rely on the ability to show data that is aggregated over times longer than a single fixation. Again, temporal-zooming can be semantic, by aggregating and summarizing data over longer time-steps, or visual, by allowing individual viewing-events to merge and blend together visually. Such visual encodings are difficult to imagine in traditional scan-paths or scarfplots as they depict direct transitions between objects at a temporal resolution of one fixation. Instead, the heatmap view in Figure X, and the reimagined scarfplot show in Figure Y can support multiple time resolutions. AOI-rivers support longer time resolutions too, but do not scale down to show individual DOI events and transitions.

\textbf{Showing attributes :} DOI attributes can be shown using conventional encoding principles such as linking attributes to visual channels (e.g., color, shape) or by relying on glyphs~\cite{maguire2012taxonomy}. 

\textbf{Reducing clutter :}
Showing many DOIs and condensing time generally results in busy visual encodings that need to be optimized to reduce clutter. First, clustering is an effective way to impose order on visualizations. Figure X shows a scarfplot that orders DOIs randomly as compared to one in which DOIs are clustered by how often there are transitions between them, a process which results in shorter scan-path lines between DOIs. Similarly, clustering by user behavior can create more organized visualizations, while clustering and arranging DOIs based on their attributes may support tasks such as [t1..t2]. Second, linking DOI appearance to their attributes can divide DOIs into visual categories or layers separable in accordance with Gestalt principles. Further dividing DOIs by how often they were viewed may achieve similar effects.  Finally, reducing the information shown using semantic zooming and DOI grouping, as described before, can also reduce clutter. 
\textbf{Supporting tasks :}
Showing multiple types of attributes concurrently for each DOI, as described previously, would make it possible to visually identify categories of DOIs, an important part of tasks described in section X, and would facilitate a visual approximation of DOI derived measures (tasks X-Y). 

Moreover, being able to visually detect clusters of DOIs with similar properties would allow analysts to detect correlations between DOI categories, times they were viewed (e.g., time tasks X,y,z), and combinations they were viewed in (e.g., transition tasks X,Y,Z). For such visual tasks to be possible however, it's not sufficient to show many DOI attributes. Additionally, DOIs need to be grouped based on when they were viewed or in what transitions they were involved. For example, scarfplots inherently group DOIs by the when they were viewed (Figure X), but the scan-path and transition matrix in Figure X and Y had to be clustered to visually group DOIs that were viewed together or around the same time. 

To support tasks X-Y[transition tasks], transition matrices should allow for  variable definitions of transitions. The mockup design in Figure X, illustrates how transitions could be specified as either the viewing of one object immediately after another, but also as the viewing of an object within a time interval of another, or within multiple transitions. Furthermore, we hypothesize that in the context of DOIs, transition graphs such as that shown in Figure X could better show DOI sequences (tasks X,Y,Z) given that node-link diagrams are significantly better at showing paths than are matrices~\cite{ghoniem2004comparison, Jianu-poster}.  Moreover, networks may directly capture different definitions of transitions visually by placing DOIs closer or farther apart, with more or less edges between them, depending on how close together DOI were viewed. 

An important factor needs to be considered when supporting analyses of multiple users (tasks X-Y). While AOI analyses often rely on few AOIs that most subjects view, DOI-instrumented visualizations can define hundreds of DOIs, which subjects, based on their interests and exploration, will only see small subsets of. These sets of viewed DOIs may differ significantly between subjects which leads to an important tradeoff: should visualizations be optimized to best show a single user's behavior, or to show a user's behavior in the context of other users' behavior? Consider the two scan-paths in Figure X. The one on the left displays the top twenty DOIs each user viewed, ordered by how much the user viewed them. The one on the right was built by considering which DOIs were viewed most by all subjects, and uses this same DOI-set and ordering for each users. The first approach shows more relevant data for each user, but makes it difficult to compare users' data. The latter shows less relevant data for each user, but the similar visual ordering makes it easy to compare user behavior. Similarly, scan-paths can be grouped by users or by DOIs as shown in Figure X. The first one makes it more clear to see [what], while the later is better for [what].


\subsection{Selection}
Selection interaction methods equip users with the capability to highlight any interesting data items to keep its track in the visualization~\cite{yi2007toward}. This method can make a visual element distinct from other elements. 

Selection have two aspects: \textit{target} and \textit{method}. First, selection target refers to visual elements which are users are able to select. For example, representation of a DOI data over multi-user eye tracking data may have selection targets such as visual elements for DOI objects, group of DOI objects, users or time windows. For example, a scanpath visualization for DOI may contain horizontal lines, one for each DOI, and super-positioned scanpath trajectories of multiple users. Selecting a DOI or group of DOIs may highlight particular horizontal lines and rectangles. Moreover, selecting a user may highlight a particular trajectory. Furthermore, a time window can be selected hence all rectangles and connections among them which fall within that time range may be highlighted. Second, selection methods are the techniques which provides users to execute selections. For DOI context, selection methods have two viewpoints: Graphical Selections and Data Selections. Graphical Selections refers the selection methods using graphical user interfaces (GUIs). Mouse clicks is the most common form of graphical selection. Moreover, it is possible to perform selection using additional GUI elements (e.g., text fields, drop-down boxes) to allow users to create queries over DOI data. Data selections refers selecting data elements rather than visual elements of the representation. Data selection is possible through selecting DOIs by identities, attributes, or by users.  Comparing with Graphical Selections, clicking a DOI-representing visual element is a selection by DOI identity in Data Selections. Selecting by attributes or by user will either be mouse click or query creation depending on complexity of representation.
 
Selection interaction can be combined with other interaction techniques. Leveraging such combinations, selection interaction supports grouping over DOI data. Users may create queries which select group of DOIs. Selecting a group of DOIs is possible by leveraging encoding interaction (discussed in sub-section~\ref{sec:Encoding}). Encoding interaction may show additional visual properties or a completely different representation for a group. Moreover, leveraging abstraction (discussed in sub-section~\ref{sec:AbstractElaborate}) may collapse visual elements of DOIs that belong to selected group. Solving task such as ``After registering a particular DOI, which DOIs does a user register'' may require group selection. Another example of such combination is using brushing and linking over two views which relate combination with connect and compare interaction (discussed in sub-section~\ref{sec:ConnectCompare}). 
	
	
\subsection{Reconfiguration}
 Reconfigurations allow users to see different perspectives of their data by modifying spatial distribution of representations~\cite{yi2007toward}.  For example, a multi-user supported scanpath may have orderings of line-trajectories by DOIs or by users. First, DOIs within scanpaths may have ordering depending on attributes either on attribute value or attribute category. Such ordering may be computed over an individual user or all users. Next, ordering by users is possible using methods such as clustering over certain DOI properties, order by user properties, or manual ordering.

\subsection{Exploration} 
Exploration interactions empower users to analyze various subset of data instances~\cite{yi2007toward}. This interaction technique is essential for over sized representations. Panning is one of the most common techniques. Coupling selection with exploration may enable users to stroll through the data from a relative point. For example, in scanpath visualization with huge number of DOIs can be explored by panning. Moreover, user can select a particular DOI and explore its changes of interest over time. 
	
\subsection{Abstraction/Elaboration} 
\label{sec:AbstractElaborate}
Abstraction/Elaboration interaction methods equip users with the capability to regulate a data representation's level of abstraction~\cite{yi2007toward}. This method facilitates to display fewer visual elements. Further, elaboration displays more visual elements. These interactions facilitate multi-level of granularity for DOI. Zooming out and in is a common form of abstraction/elaboration. Displaying more or less data over DOI data is possible using few techniques. First, adjusting the space allotted to each user data may facilitate showing as much data as possible for each user data. Second, by specifying how much data to show can control this interaction. Third, controlling time scales may reveal or hide data on visualization such as AOI-rivers, scarfplots, or heatmaps. Third, computing hierarchy among DOIs enables collapsing and expanding within visualizations which is supported in most of the AOI-based visualizations. Finally, details on demand interaction can be used to achieve such interaction (e.g. mouse over DOIs or users).
	
	
\subsection{Filtering}
\label{sec:Filtering}

  Filtration interactions enable users to display visual elements based on some explicit conditions~\cite{yi2007toward}. For example, DOIs are defined with an attribute $score_{gaze}$ where $0 \leq score_{gaze} \leq 1$. Using filtering techniques, user can regulate a threshold $T$ such as $score_{gaze} \geq T$. Assigning value closed to the maximum value for $T$ ( e.g. $T =0.85$) would result relatively tiny number of DOIs. Additionally, for temporal aspect of DOI data, filtering is possible by specifying time frames. 	

	
\subsection{Connection/Comparison}
\label{sec:ConnectCompare}
 Yi et al. defines connections as the interaction technique that is used for indicating affiliations between represented data items, and displaying dormant data items that are important to a particular item~\cite{yi2007toward}.  We include comparison as a separate design consideration, as comparison is essential in experimental studies and thus in analyses of eye-tracking data. Connection can also be coupled with selection. For example, in a scanpath visualization when user selects a perpendicular line for a DOI, the related DOI perpendicular lines get highlighted. Multiple views over same data can be displayed using juxtaposition, super-positions, and explicit encodings~\cite{gleicher2011visual}. Brushing and linking over juxtaposed views can connect DOI objects. Moreover, using super-position and explicit encoding can be relatively hard to connect DOI objects. 

