\section{A visual design space for DOI analysis}
The previous sections emphasize the differences between AOI and DOI data and provide an intuition that visual methods that have been developed for AOI analysis do not automatically extend to DOI data. Broadly, AOI visualizations such as scanpaths, scarfplots , and AOI-rivers cannot handle the scale of DOI data, which is captured at high granularities and over extended periods of time. To address this, novel visualizations will need to allow analysts to explore DOI data at multiple levels of detail, including a range of temporal scales (e.g., several seconds vs. several minutes) and data granularities (e.g., individual data objects vs. classes of objects).  

AOI visualizations were also not designed to show or leverage the attributes that are typical of DOIs.  Attributes need to be shown visually and queries flexibly to allow analyzers to identify links between users' interest in data and that data's properties. They can also be leveraged to help deal with the scale of the data by allowing users to filter, highlight, and aggregate data with specific attribute configurations.

Finally, collecting users' interest in richly annotated DOIs over hour long analysis and from many users can support analytic workflows different from those typical of AOI analyses. For example, analysts may care less about individual DOIs and more about groups of DOIs defined by their attributes and they may explore larger temporal scales rather than individual viewing transitions between DOIs. The task taxonomy described in Section~\ref{sec:Taxonomy} illustrate such tasks, and visual methods should be geared towards supporting them. 

In the following sections we explore a design space for visual DOI analysis in more detail. We hypothesize that given the scale and complexity of DOI data, interaction will play a significant role in analyzing it. As such, we treat both visual encoding and interaction design in equal parts. Moreover, our goal is not to propose novel visual encodings, which would be a research effort in its own right, but to discuss general design principles that apply to visual DOI data analyses. As such, we considered it practical to ground our discussion by using variations of AOI visualizations to exemplify what functionality needs to be implemented and how it could be implemented. 

To ensure that our discussion is exhaustive and captures all design dimensions, we organize it by Yi's taxonomy of interaction tasks~\cite{yi2007toward}. Yi et al. proposed seven categories of interactions based on the concept of user's aim to a visual task: Selection, Encoding, Reconfiguration, Exploration, Abstraction/Elaboration, Filtering, and Connection and comparison~\cite{yi2007toward}. 

\subsection{Encoding}
Encoding interaction methods facilitate users to modify each data element's visual representation including visual encoding variables such as color, size, and shape~\cite{yi2007toward}. Blascheck et al. listed visual encoding for AOI data such as scanpaths, scarf-plots, rivers and transition matrices~\cite{blascheck2014state}. DOI data is a highly granular and annotated AOI data at core. Hence, AOI visualization encodings may be appropriate to DOI data. We hypothesize that, visual encodings for AOI data is insufficient for DOI data. First, visual encodings are required to support high scalability of DOI. For example, DOI data collected from an eye-tracking study with hour-long sessions, more than ten users, and highly granulated DOI definition will end up having thousands of DOI units. Visual encoding such as AOI-river will be unsuitable for such cases as cardinality of AOIs is a limitation~\cite{burch2013aoi}. Second, displaying and leveraging attributes are required to visual encodings. Finally, visual encodings required to support some tasks which is not possible in AOIs. AOI techniques can be applied in all cases directly and augmented by interactions. For example, hovering over a scarfplot items could show details in textual forms which is not reasonable. Moreover, the encoding itself should support some of the tasks. For example, multiple objects viewing at the same time is possible either because we do not know what a user looks at or because we are aggregating over larger time steps.

Some modification may be applicable to DOI visual encodings. We discuss some of them below. 
\begin{itemize}
	\item \textbf{Showing identity :} Showing identity of visual elements will facilitate representing DOI data. Although, some AOIs visualizations already do that (e.g., AOI-river) but some do not (e.g., scarfplot) resulting requirement of modifications. For example in context of education domain, an article may have hundreds of unique words. If we want to solve the task \textit{``Which user looked at the word \textbf{Algorithm} for the maximum time?''} (exemplified from \textit{``Which user has registered to a particular DOI?''}) then the representation for it will require showing identity of DOIs. A Scarplot with a colored band per DOI will be insufficient to denote DOIs. If we assign a unique color to each DOI as a unique word in this case then, it will end up having similar color for two different DOIs. A possible modification can be done by adding a balloon with labels of DOIs attached to respective bands in the scarfplot. 
	\item \textbf{Displaying attributes :} Simple textual overlays might be insufficient to display multiple attributes.  Some encoding variables can be modified to display attributes such as color, shapes. Moreover, visual encodings may employ glyphs to display attributes. For example in context of movie domain, DOIs defined over movies will have multiple attributes such as title, star rating, genre, release date. Solving task such as \textit{How many movies user1 looked at having same genre and same rating of \textbf{Star Wars}?} (exemplified from from \textit{How many DOIs are registered which are connected over DOI properties of a particular DOI by a particular user?}). A scarfplot representing such scenario with colored bands per DOI, may have colors representing genres of movies but allocating rating attributes is hard. Attaching color bands with glyphs (e.g., star shapes with rating numbers inside) may be a solution for it.
	\item \textbf{Visual clustering support :} Visual elements often have connections among them. A representation of such case becomes jumbled if visual elements are spatially dispersed. Using clustering algorithms for positioning visual elements with their connected elements will significantly decrease visual clutters. For example in a scanpath visualization, a horizontal line represent a DOI and disjointed rectangles on this line represent times where that DOI was registered. Rectangles on different horizontal lines are connected representing chronological trajectories of DOI registrations. If DOIs with higher connections between them are placed distant will endup having lengthy trajectory lines. Clustering algorithms can effectively calculate DOIs sequences hence DOIs with higher connections will end up positioned nearby. Moreover, representing a multi-user data in a scarfplot can be accomplished by juxtaposing single-user scarfplots. Scarfplots with representations of attributes (e.g., balloons attached with bands) will significantly increase visual clutters. If similar user data representations are juxtaposed then it is possible to use a single attribute representation for multiple bands. Clustering algorithms can calculate such user-data sequences over user-data similarities hence decreasing significant amount of visual clutters.  
	\item \textbf{Showing only ``top'' data :} We found that interest in data decays exponentially with its relevance to a task; as such, not all DOIs are equal. Some DOIs will be viewed for only a fraction of a second, while others will be viewed multiple times for many seconds. 
	\item \textbf{Showing semantic filtering/zooming :} Granularity of DOIs can be negotiated using semantic filtering or zooming by showing more or less data depending on the zoom level.
	\item \textbf{Support grouping:} Grouping is possible either by grouping DOIs by attribute or grouping users by calculating similarity index among them over DOI data. Moreover, some encodings (e.g. AOI river, transition matrices) exist for aggregated users by considering all users as a single user via condensing and aligning. 
\end{itemize}

\subsection{Selection}
Selection interaction methods equip users with the capability to highlight any interesting data items to keep its track in the visualization~\cite{yi2007toward}. This method can make a visual element distinct from other elements. 

Selection have two aspects: \textit{target} and \textit{method}. First, selection target refers to visual elements which are users are able to select. For example, representation of a DOI data over multi-user eye tracking data may have selection targets such as visual elements for DOI objects, group of DOI objects, users or time windows. For example, a scanpath visualization for DOI may contain horizontal lines for each DOI and super-positioned scanpath trajectories of multiple users. Selecting a DOI or group of DOIs may highlight particular horizontal lines and rectangles. Moreover, selecting a user may highlight a particular trajectory. Furthermore, a time window can be selected hence all rectangles and connections among them which fall within that time range may be highlighted. Second, selection methods are the techniques which provides users to execute selections. Selection is possible within DOI by identity, attribute, or by user. Again considering the scanpath example discussed above, users may click on horizontal lines to select DOIs by identity or click on trajectories to select by user. Moreover, users may create queries over DOI attributes to select multiple DOIs. 

Selection interaction can be combined with other interaction techniques. For example, implementing selection over brushing and linking across multiple data views or user datasets, supports comparison tasks. 

	
	
\subsection{Reconfiguration:} Reconfigurations allow users to see different perspectives of their data by modifying spatial distribution of representations~\cite{yi2007toward}. For example, a multi-user supported scanpath may have orderings of line-trajectories by DOIs or by users. First, DOIs within scanpaths may have ordering depending on attributes either on attribute value or attribute category. Such ordering may be computed over an individual user or all users. Next, ordering by users is possible using methods such as clustering over certain DOI properties, order by user properties, or manual ordering.

\subsection{Exploration:} Exploration interactions empower users to analyze various subset of data instances~\cite{yi2007toward}. This interaction technique is essential for over sized representations. Panning is one of the most common techniques. Coupling selection with exploration may enable users to stroll through the data from a relative point. For example, in scanpath visualization with huge number of DOIs can be explored by panning. Moreover, user can select a particular DOI and explore its changes of interest over time. 
	
\subsection{Abstraction/Elaboration :} Abstraction/Elaboration interaction methods equip users with the capability to regulate a data representation's level of abstraction~\cite{yi2007toward}. This method facilitates to display fewer visual elements. Further, elaboration displays more visual elements. These interactions facilitate multi level of granularity for DOI. Zooming out and in is a common form of abstraction/elaboration. Displaying more or less data over DOI data is possible using few techniques. First, adjusting the space allotted to each user data may facilitate showing as much data as possible for each user data. Second, by specifying how much data to show can control this interaction. Third, controlling time scales may reveal or hide data on visualization such as AOI-rivers, scarfplots, or  heatmaps. Third, computing hierarchy among DOIs enables collapsing and expanding within visualizations which is supported in most of the AOI-based visualizations. Finally, details on demand interaction can be used to achieve such interaction (e.g. mouse over DOIs or users).
	
	
\subsection{Filtering} Filtration interactions enable users to display visual elements based on some explicit conditions~\cite{yi2007toward}. For example, DOIs are defined with an attribute $score_{gaze}$ where $0 \leq score_{gaze} \leq 1$. Using filtering technique, user can regulate a threshold $T$ such as $score_{gaze} \geq T$. Assigning value closed to the maximum value for $T$ ( e.g. $T =0.85$) would result relatively tiny number of DOIs. Additionally, for temporal aspect of DOI data, filtering is possible by specifying time frames. 	

	
	\subsection{Connection and comparison} Yi et al. defines connections as the interaction technique that is used for indicating affiliations between represented data items, and displaying dormant data items that are important to a particular item~\cite{yi2007toward}. We include comparison as a separate design consideration, as comparison is essential in experimental studies and thus in analyses of eye-tracking data. Connection can also be coupled with selection. For example, in a scanpath visualization when user selects a perpendicular line for a DOI, the related DOI perpendicular lines get highlighted. Multiple views over same data can be displayed using juxtaposition, super-positions, and explicit encodings~\cite{gleicher2011visual}. Brushing and linking over juxtaposed views can connect DOI objects. Moreover, using super-position and explicit encoding can be relatively hard to connect DOI objects. 
