\section{Visual Methods}
In this section, we discuss potential visualization techniques to support tasks mentioned in Section~\ref{sec:Taxonomy}. Blascheck et al. demonstrated some AOI-based visualization techniques such as Scanpaths, Scarfplot, AOI-Rivers and others~\cite{blascheck2014state} considering temporal aspects of data and relations among AOIs. AOI-based visualization techniques are ineffective to DOI data because these visualizations may not entirely support visualizing DOI data due to DOI's inherent properties such as high scalability, multi-level granularity, multiple-users, and multiple-attributes. First, Tuning temporal resoultion of time-spans may result huge amount of DOIs. Second, granularity of the DOI on object level may increase amount of DOIs. For example, if DOIs are defined over individual objects rather group of objects will result copious amount of DOIs. Third, multiple-user DOI data contains more DOIs than single-user DOI data. Finally, AOI-based visualizations generally support few attributes of AOIs. DOIs may have ample attributes which may not support AOI-based visualizations. For example, A scanpath diagram shows trajectory of AOI fixations over time where a vertical or horizontal line represent an AOI. Segments of that line represent time-spans. Thus, applying scanpaths to visualize DOI data with ample amount of DOIs, may result cluttered or high-dimension images.  Moreover, numerous properties per DOI is not supported to scanpaths. 

To address these issues, adjusting representations and incorporating interactions are required in AOI-based visualizations. Gleicher et al. proposed three basic categories of visual designs: juxtapositions, super-positions, and explicit encodings~\cite{gleicher2011visual}. Multiple-user DOI data can be visualized by juxtaposing single-user visualizations. This approach may be ineffective for huge amount of users. Explicit encoding may still be ineffective because multi-user data may not have any relationship among them. Applying super-positions may resolve this issue but it may create new issues such as visual clutters, monotony. Using colors and clustering DOIs may be effective to these issues. Hence, a single representation can not be universally effective on all scenarios. A particular representation or a hybrid representation may be applicable depending on the scenario. 

Interactions over a visualization permits to choose appropriate representations. Yi et al. proposed seven categories of interactions based on the concept of user's aim to a visual task: Encoding, Reconfiguration, Filtration, Exploration, Selection, Abstraction/Elaboration, and Connection~\cite{yi2007toward}. 

\begin{itemize}
	\item \textbf{Encoding:} Encoding interaction triggers a visualization morph to a new representation. This interaction enables users to choose among different representation discussed above. Moreover, encoding interaction is used to change shape, size, and color of a visual element. For example, scanpath representation of a multi-user DOI data can have vertically juxtaposed form and super-positioned form. A user can choose either of the representation using a drop-down list, based on the number of subject the data contains. Moreover, label-colors, size, and shape of the time-span mark can be changed using encoding interaction. 
	\item \textbf{Reconfiguration:} Reconfiguration interaction enables arranging visual elements to a different positions. For example, a scanpath with huge number of DOIs in a small area will result visual clutters where perpendicular lines dedicated to DOIs will be adjoining to adjacent lines. Tuning the distance between lines can reduce visual clutters in the visualization. 
	\item \textbf{Filtration:} Decreasing DOI count in a visualization is a significant requirement. Filtration interaction enables users to display visual elements conditionally. For example, DOIs are defined with an attribute $score_{gaze}$ where $0 \leq score_{gaze} \leq 1$. Using filtration technique, user can regulate a threshold $T$ such as $score_{gaze} \geq T$. Assigning  value closed to the maximum value for $T$ ( e.g. $T =0.85$) would result relatively tiny number of DOIs. 
	\item \textbf{Selection:} Selection interaction is used to make visual element relatively more salient. Selection interaction can make a visual element distinct from other elements. For example, in a scanpath visualization, user can click a perpendicular line area to select a DOI to track changes to its interest on subjects over time. 
	\item \textbf{Exploration:} Exploration interaction is essential for over sized representations. Panning is one of the most common techniques. Coupling selection with exploration may enable users to stroll through the data from a relative point. For example, in scanpath visualization with huge number of DOIs can be explored by panning. Moreover, user can select a particular DOI and explore its changes of interest over time. 
	\item \textbf{Abstraction/Elaboration:} Abstraction interaction facilitates to display fewer visual elements. Further, elaboration displays more visual elements. Zooming out and in is a common form of abstraction/elaboration. For example, DOIs with multiple attributes can be displayed with fewer attributes while zoomed out and more attributes while zoomed in. 	
	\item \textbf{Connection:} Connection interactions shows related items of visual elements. Connection can also be coupled with selection. For example, in a scanpath visualization when user selects a perpendicular line for a DOI, the related DOI perpendicular lines get highlighted. 
\end{itemize}



