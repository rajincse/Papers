\section{Introduction}
Eye-tracking devices report users' gaze positions on a screen. Typical eye-tracking user studies record gaze positions of subjects performing visual tasks. Eye-tracking has broad usage as a diagnostic tool in psychology, cognitive sciences, human-computer interaction, and visualization research~\cite{duchowski2002breadth}.

Traditionally, eye-tracking data is analyzed offline by human analysts using one of two approaches: point-based analysis and Area of Interest (AOI) -based analysis~\cite{blascheck2014state}. Point-based techniques treat fixations as individual units of analysis and can reveal the distribution or sequence of fixations over visual stimuli (e.g. gaze heatmaps, scanpaths). AOI-based techniques rely on a preliminary annotation of visual stimuli with AOIs, image regions of interest with particular semantic meaning.  Both approaches require significant data annotation and interpretation efforts and are laborious in studies involving multiple users, long sessions, and interactive stimuli. 

Recently, Data of Interest (DOI) analysis of eye-tracking data was proposed to alleviate these limitations. When visual content is dynamically generated, such as for data visualizations, the structures of the visual content is known at rendering time.  In such cases, the source code can be instrumented to match gaze coordinates from an eye-tracker to visual items rendered by the visualization on the screen, and implicitly to the underlying data, in real-time. For example, the code of a node-link visualization can be instrumented to correlate gaze-coordinates to positions of individual nodes and automatically report which nodes subjects are viewing at any time.  

Many visualizations are generated from data antecedent to rendering. Moreover, many visualizations contains interactivity and animations. We hypothesize, analysis of eye-tracking data in data-space is more effective than traditional point-based and AOI-based analyses. While DOIs are essentially dynamically computed and managed AOIs, they are inherently different in granularity, scale, and semantics from AOI data. Previously, DOI has not been explored exhaustively in data visualization. Moreover, facilitating DOI in eye-tracking data analysis require formal definitions and protocols. In this paper, we formalize DOI in context of eye-tracking data analysis. Moreover, we compare DOI-based analysis with existing point-based and AOI-based analysis.   

Our contributions are: (i) a formal definition of DOI analysis, data model and examples of DOI in multiple domains (Section X1) ;( ii) a task-taxonomy for DOI-based visualization (Section X2); (iii) a demonstration of existing visualization techniques in context of DOI-based visualization (Section X3);(iv) limitations of using DOI analysis (Section X4). 

