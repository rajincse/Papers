\subsection{RelatedWork}
Eye tracking is an effective mechanism to inspect people�s visual patterns and attention processes [11], and is widely used is fields such as psychology, neuroscience, human-computer interaction, and data visualization [10] to investigate how people perceive, react to, and think with visual stimuli. For example, eye-tracking experiments were undertaken to study how people recognize faces [12], attention pattern linked with physical discomforts [13], and how students learn graphical contents [14]. Moreover, contributions using eye-tracking within data visualization field include, among others, studies on graph readability [15, 16, 17], tree-drawing perception [18, 19], and visualization evaluation [20].  The breadth and scope of eye tracking research  is growing rapidly as eye-trackers become increasingly  accurate, fast, and affordable.  