\section{Related Work}
Eye tracking is an effective mechanism to inspect people's visual patterns and attention processes~\cite{jacob1991use}, and is widely used is fields such as psychology, neuroscience, human-computer interaction, and data visualization~\cite{duchowski2002breadth} to investigate how people perceive, react to, and think with visual stimuli. In a traditional eye-tracking user study, a human subject performs visual tasks in front of a computer screen attached with an eye-tracking device. For example, eye-tracking experiments were undertaken to study how people recognize faces~\cite{guo2014perceiving}, attention pattern linked with physical discomforts~\cite{vervoort2013attentional}, and how students learn graphical contents~\cite{mayer2010unique}. Moreover, contributions using eye-tracking within data visualization field include, among others, studies on graph readability~\cite{pohl2009comparing, huang2008beyond, huang2005people}, tree-drawing perception~\cite{burch2011evaluation, burch2013visual}, and visualization evaluation~\cite{kim2012does}.  The breadth and scope of eye tracking research is growing rapidly as eye-trackers become increasingly accurate, fast, and affordable.  

Analyzing eye-tracking data is challenging. Generally, eye-tracking data is analyzed by two approaches: point-based and area-of-interest (AOI)-based~\cite{blascheck2014state}. Point-based analyses dwell on the spatial and temporal distribution of gaze samples or fixations, and aim to capture the overall movement of subjects' gaze over the stimulus.  While  annotations are not required in this method, experimenters generally need to interpret data and relate it to the semantic content of the stimuli by using visualizations such as heatmap~\cite{mackworth1958eye} or scanpaths~\cite{noton1971scanpathsA, noton1971scanpathsB}. For extended studies and many users such visualizations can be cluttered and difficult to use. In AOI-based analyses, annotated regions with high importance to a hypothesis (AOIs) are defined over a stimulus. Traditionally, analysts defines AOIs manually over stimuli. This can be difficult for dynamic or interactive stimuli as AOIs need to be redefined or reshaped whenever a stimulus changes. However, automated AOI definition utilizing gaze-clustering algorithms exist~\cite{privitera2000algorithms}. Andrienko et al.~\cite{andrienko2012visual} proposed visual analytics methodologies for inspecting eye-tracking data on techniques referring visualization and statistical aspect of the data. Recently, Alam et al. proposed analysis of eye-tracking data-space:Data of Interest (DOI)-based analysis~\cite{alamdata}. Analogous to AOIs, DOIs are part of data used to generate stimuli which can be highly granular, albeit there is no formal definition, and task-taxonomy exist for DOI. Unlike AOIs, DOIs are created for the data-model that a visualiztion serves and their specification is tied to the rendering code of the visualizations. Once a visualization instrumented, DOI can be obtained automatically for any new data set, and for any sequence of interactions. If a DOI is not shown on the screen, no data will be recorded for that DOI.

 In context of data visualization, Shneiderman proposed a task by data taxonomy~\cite{shneiderman1996eyes}. Moreover, Brehmer et al. recently proposed multi-level typology that can be aided to create a complete task description~\cite{brehmer2013multi}. Amar et al. proposed ten low-level visual-analytics tasks types as questions experimenters may ask while performing experiments with tabular data~\cite{amar2005low}. Finally, task taxonomies exist based on the category of visualizations, such as graph visualization~\cite{lee2006task}, group-level graph~\cite{saket2014group}, and multidimensional data visualization~\cite{ward2002taxonomy}. 

