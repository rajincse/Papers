\begin{abstract}
Eye-tracking data is traditionally analyzed by looking at where on a visual stimulus subjects fixate, or by using area-of-interests (AOI) defined onto visual stimuli to group semantically and facilitate more advanced analyses. Recently, there is increasing interest in methods that look at what users are looking rather than where they are. By instrumenting the code that transforms a data model into visual content, gaze coordinates reported by an eye-tracker can be mapped directly to granular data shown on the screen, producing sequences of data objects that subjects viewed in an experiment. Such data collection, which is called gaze to object mapping (GTOM) or data-of-interest (DOI) analysis, can be done reliably with limited overhead and can facilitate research workflows not previously possible. Our paper contributes to establishing a foundation of DOI analyses. To this end we define a DOI data model and highlight its differences to AOI data in structure and scale; we define and exemplify a taxonomy of DOI enabled tasks; and we discuss the space of visual designs that can support this taxonomy.
\end{abstract}
