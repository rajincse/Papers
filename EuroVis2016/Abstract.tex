\begin{abstract}
   Eye-tracking user study data can be analyzed using visualization techniques. Generally, visual analysis of eye-tracking data is performed with the help of superimposing heatmaps or scanpaths or defining area-of-interests (AOI) manually. Recently, eye-tracking data analysis in data space: data-of-interest (DOI)-based analysis is proposed. In a DOI-based analysis, specifying highly granular data elements called DOIs are required to be defined over stimuli-generating data which can be instrumented with source code of visualization. We formally define DOI alongside defining its data model and task list relevant to DOI-based analysis. We articulate examples of DOIs over  data-sets for visualization domains. Moreover, we discuss limitations of using DOI-based analysis. 
\end{abstract}
