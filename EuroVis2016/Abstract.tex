\begin{abstract}
Eye-tracking data is traditionally analyzed by looking at where on a visual stimulus subjects fixate, or by using area-of-interests (AOI) defined onto visual stimuli to facilitate more advanced analyses. Recently, there is increasing interest in methods that capture what users are looking rather than where they are looking. By instrumenting visualization code that transforms a data model into visual content, gaze coordinates reported by an eye-tracker can be mapped directly to granular data shown on the screen, producing temporal sequences of data objects that subjects viewed in an experiment. Such data collection, which is called gaze to object mapping (GTOM) or data-of-interest analysis (DOI), can be done reliably with limited overhead and can facilitate research workflows not previously possible. Our paper contributes to establishing a foundation of DOI analyses by defining a DOI data model and highlighting its differences to AOI data in structure and scale; by defining and exemplifying a taxonomy of DOI enabled tasks; and by exploring the space of visual designs that can support this taxonomy.
\end{abstract}
