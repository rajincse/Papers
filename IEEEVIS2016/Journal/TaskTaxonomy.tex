\section{A DOI Task Taxonomy}
\label{sec:Taxonomy}

We present a task taxonomy for DOI analyses. We used Amar et al.'s taxonomy of visualization tasks~\cite{amar2005low} to explore questions that DOI data answers. We show a representative sample of tasks in Figure~\ref{fig:taxonomy}, and exemplify them using the data, and domains shown in Table~\ref{tab:ExampleDOI}.

Two principles emerged while exploring DOI tasks. These principles allow us to consider novel DOI tasks by generalizing from familiar AOI tasks. First, while AOIs are typically identified by their identity (e.g., $AOI_1$), DOIs should be identified by constraints on their attributes (e.g., $DOI$ (type:`movie', name:`Dark Knight'). Also, tasks targeting individual DOIs should extend to selections of DOIs (e.g., DOI(type:`movie', $rating>8$)). For brevity, we will refer to such selections as DOI categories. Familiar AOI questions such as ``What $AOI$ has a user viewed at time $t$?'', ``When was $AOI_1$ viewed?'', and ``How prevalent is a transition from $AOI_1$ to $AOI_2$?'', then become ``What are the attributes of the $DOI$ viewed at time $t$?'', ``When was a DOI category viewed?'', and ``How prevalent are transitions between two DOI categories?''.

Second, analysis should be possible at multiple levels of detail. Analysts should be able to explore user behavior at time scales of seconds, minutes, or hours, and query for individual DOIs or groups of DOIs. DOIs support such tasks, since their attributes can be used to answer questions about categories of data viewed during specific intervals (e.g., ``Did a user look more at comedies or dramas?'', ``What is the average rating of movies viewed during an analysis?''). All DOI tasks should be possible for one or more subjects, and for flexibly defined time intervals (e.g., ``first half of the user's analysis'', ``the last five minutes of analysis'').

After considering a range of tasks, we decided to divide them into four broad categories. ``Summarize'' tasks are aimed at capturing what types of data subjects were interested in during the whole or part of an experiment, and at computing derived values from attributes of viewed DOIs.  ``Type of data over time'' questions reveal temporal patterns in users' data interests. As noted previously, such tasks should support multiple time-scales and time intervals. ``DOI transitions and sequences'' tasks are intended to facilitate insights into how DOIs are viewed together. In line with the desideratum for multi-granular analyses, such tasks should go beyond exploring just direct transitions between pairs of DOIs, and instead look at broader groups and sequences of co-viewed DOIs. Finally, ``Compare, contrast, and correlate'', tasks allow analysts to compare either groups of users (e.g., novice vs. novice users), or time intervals (e.g., early in an analysis vs. late in an analysis).

To further ground our results, Figure 2 relates our taxonomy to Amar et al.'s ten generic visualization task categories: Retrieve Value, Filter, Derive Value, Find Extremum, Sort, Determine Range, Characterize Distribution, Find Anomalies, Cluster, Correlate.
