\documentclass[journal]{vgtc}                % final (journal style)
%\documentclass[review,journal]{vgtc}         % review (journal style)
%\documentclass[widereview]{vgtc}             % wide-spaced review
%\documentclass[preprint,journal]{vgtc}       % preprint (journal style)
%\documentclass[electronic,journal]{vgtc}     % electronic version, journal

%% Uncomment one of the lines above depending on where your paper is
%% in the conference process. ``review'' and ``widereview'' are for review
%% submission, ``preprint'' is for pre-publication, and the final version
%% doesn't use a specific qualifier. Further, ``electronic'' includes
%% hyperreferences for more convenient online viewing.

%% Please use one of the ``review'' options in combination with the
%% assigned online id (see below) ONLY if your paper uses a double blind
%% review process. Some conferences, like IEEE Vis and InfoVis, have NOT
%% in the past.

%% Please note that the use of figures other than the optional teaser is not permitted on the first page
%% of the journal version.  Figures should begin on the second page and be
%% in CMYK or Grey scale format, otherwise, colour shifting may occur
%% during the printing process.  Papers submitted with figures other than the optional teaser on the
%% first page will be refused.

%% These three lines bring in essential packages: ``mathptmx'' for Type 1
%% typefaces, ``graphicx'' for inclusion of EPS figures. and ``times''
%% for proper handling of the times font family.

\usepackage{mathptmx}
\usepackage{graphicx}
\usepackage{times}
\usepackage{amsmath}
\usepackage{ulem}

%% We encourage the use of mathptmx for consistent usage of times font
%% throughout the proceedings. However, if you encounter conflicts
%% with other math-related packages, you may want to disable it.

%% This turns references into clickable hyperlinks.
\usepackage[bookmarks,backref=true,linkcolor=black]{hyperref} %,colorlinks
\hypersetup{
  pdfauthor = {},
  pdftitle = {},
  pdfsubject = {},
  pdfkeywords = {},
  colorlinks=true,
  linkcolor= black,
  citecolor= black,
  pageanchor=true,
  urlcolor = black,
  plainpages = false,
  linktocpage
}

%% If you are submitting a paper to a conference for review with a double
%% blind reviewing process, please replace the value ``0'' below with your
%% OnlineID. Otherwise, you may safely leave it at ``0''.
\onlineid{0}

%% declare the category of your paper, only shown in review mode
\vgtccategory{Research}

%% allow for this line if you want the electronic option to work properly
\vgtcinsertpkg

%% In preprint mode you may define your own headline.
%\preprinttext{To appear in an IEEE VGTC sponsored conference.}

%% Paper title.

\title{A Taxonomy of Tasks and Visual Solutions for Data of Interest (DOI) Eye-Tracking Analyses}

%% This is how authors are specified in the journal style

%% indicate IEEE Member or Student Member in form indicated below
% for anonymous conference submission please enter your SUBMISSION ID
% instead of the author's name (and leave the affiliation blank) !!
%\author[D. Fellner \& S. Behnke]
       %{D.\,W. Fellner\thanks{Chairman Eurographics Publications Board}$^{1,2}$
        %and S. Behnke$^{2}$
%%        S. Spencer$^2$\thanks{Chairman Siggraph Publications Board}
        %\\
%% For Computer Graphics Forum: Please use the abbreviation of your first name.
         %$^1$TU Darmstadt \& Fraunhofer IGD, Germany\\
         %$^2$Institut f{\"u}r ComputerGraphik \& Wissensvisualisierung, TU Graz, Austria
%%        $^2$ Another Department to illustrate the use in papers from authors
%%             with different affiliations
       %}
\author{Submission XXX}
%\authorfooter{
%%% insert punctuation at end of each item
%\item
 %Roy G. Biv is with Starbucks Research. E-mail: roy.g.biv@aol.com.
%\item
 %Ed Grimley is with Grimley Widgets, Inc.. E-mail: ed.grimley@aol.com.
%\item
 %Martha Stewart is with Martha Stewart Enterprises at Microsoft
 %Research. E-mail: martha.stewart@marthastewart.com.
%}

%other entries to be set up for journal
%\shortauthortitle{Biv \MakeLowercase{\textit{et al.}}: Global Illumination for Fun and Profit}
%\shortauthortitle{Firstauthor \MakeLowercase{\textit{et al.}}: Paper Title}

%% Abstract section.
\abstract{
Eye-tracking data is traditionally analyzed by looking at where on a visual stimulus subjects fixate, or by using area-of-interests (AOI) defined onto visual stimuli to facilitate more advanced analyses. Recently, there is increasing interest in methods that capture what users are looking rather than where they are looking. By instrumenting visualization code that transforms a data model into visual content, gaze coordinates reported by an eye-tracker can be mapped directly to granular data shown on the screen, producing temporal sequences of data objects that subjects viewed in an experiment. Such data collection, which is called gaze to object mapping (GTOM) or data-of-interest analysis (DOI), can be done reliably with limited overhead and can facilitate research workflows not previously possible. Our paper contributes to establishing a foundation of DOI analyses by defining a DOI data model and highlighting its differences to AOI data in structure and scale; by defining and exemplifying a taxonomy of DOI enabled tasks; and by exploring the space of visual designs that can support this taxonomy.
} % end of abstract

%% Keywords that describe your work. Will show as 'Index Terms' in journal
%% please capitalize first letter and insert punctuation after last keyword
%\keywords{Radiosity, global illumination, constant time}

%% ACM Computing Classification System (CCS). 
%% See <http://www.acm.org/class/1998/> for details.
%% The ``\CCScat'' command takes four arguments.

%\CCScatlist{ % not used in journal version
 %\CCScat{K.6.1}{Management of Computing and Information Systems}%
%{Project and People Management}{Life Cycle};
 %\CCScat{K.7.m}{The Computing Profession}{Miscellaneous}{Ethics}
%}

%% Uncomment below to include a teaser figure.
  %\teaser{
 %\centering
 %\includegraphics[width=16cm]{CypressView}
  %\caption{In the Clouds: Vancouver from Cypress Mountain.}
  %}

%% Uncomment below to disable the manuscript note
%\renewcommand{\manuscriptnotetxt}{}

%% Copyright space is enabled by default as required by guidelines.
%% It is disabled by the 'review' option or via the following command:
% \nocopyrightspace

%%%%%%%%%%%%%%%%%%%%%%%%%%%%%%%%%%%%%%%%%%%%%%%%%%%%%%%%%%%%%%%%
%%%%%%%%%%%%%%%%%%%%%% START OF THE PAPER %%%%%%%%%%%%%%%%%%%%%%
%%%%%%%%%%%%%%%%%%%%%%%%%%%%%%%%%%%%%%%%%%%%%%%%%%%%%%%%%%%%%%%%%

\begin{document}

%% The ``\maketitle'' command must be the first command after the
%% ``\begin{document}'' command. It prepares and prints the title block.

%% the only exception to this rule is the \firstsection command
%\firstsection{Introduction}

\maketitle

\begin{abstract}
   Eye-tracking user study data can be analyzed using visualization techniques. Generally, visual analysis of eye-tracking data is performed with the help of superimposing heatmaps or scanpaths or defining area-of-interests (AOI) manually. Recently, eye-tracking data analysis in data space: data-of-interest (DOI)-based analysis is proposed. In a DOI-based analysis, specifying highly granular data elements called DOIs are required to be defined over stimuli-generating data which can be instrumented with source code of visualization. We formally define DOI alongside defining its data model and task list relevant to DOI-based analysis. We articulate examples of DOIs over  data-sets for visualization domains. Moreover, we discuss limitations of using DOI-based analysis. 
\end{abstract}
\section{Introduction}

Eye-trackers can tell us where on the screen computer users are looking, and have been used extensively as diagnostic tools in disciplines such as psychology, cognitive science, human-computer interaction, and Visualization research~\cite{duchowski2002breadth}. Traditional eye-tracking analyses, such as point-based analysis and area of interest (AOI) analysis~\cite{blascheckstate}, rely on gaze coordinates collected in conjunction with rendered visual stimuli, and require a significant amount of manual input from human annotators to relate gazes to the semantic meaning of the stimuli~\cite{alamdata}. Such efforts meant that eye-tracking analyses could only be performed relatively laboriously, offline. 

While this may have been acceptable while eye-trackers were expensive and data relatively difficult to collect, accesible eye-trackers (e.g., Tobii EyeX) open up novel analysis possibilities. One of these is analyzing eye-tracking data in real-time. For example, it is nowadays conceivable that classroom computers could be equipped with eye-trackers, and visual learning environments instrumented to capture in detail what learning concepts students are looking at. Such data could be used in real time by instructors, to identify students that don’t tend to concepts known to be important or fail to make progress, and provide proactive help. 

Recently, Alam et al.~\cite{alamdata} proposed the data of interest (DOI) analysis of eye-tracking data. This approach simplifies the collecting and analysis of eye tracking data to the point that it can be performed in real time. It involves instrumenting the rendering code of a visualization so that gazes  are automatically related to the visual content that is displayed on the screen. DOI-based analysis of eye-tracking data works on the data-space instead of image space. DOI-based analysis method can be instrumented in the visualization system. The method provides fuzzy scores of elements which viewers are probably put their interest. 

We demonstrate our method of automated information retrieval from eye-tracking user study data we conducted a meta user study over the user study data-set from Alam et al.~\cite{alamdata}. More in-depth description of the meta user study is provided in Section~\ref{sec:Evaluation}. 
 

\section{Related Work}
Eye tracking is an effective mechanism to inspect people's visual patterns and attention processes~\cite{jacob1991use}, and is widely used is fields such as psychology, neuroscience, human-computer interaction, and data visualization~\cite{duchowski2002breadth} to investigate how people perceive, react to, and think with visual stimuli. In a traditional eye-tracking user study, a human subject performs visual tasks in front of a computer screen attached with an eye-tracking device. For example, eye-tracking experiments were undertaken to study how people recognize faces~\cite{guo2014perceiving}, attention pattern linked with physical discomforts~\cite{vervoort2013attentional}, and how students learn graphical contents~\cite{mayer2010unique}. Moreover, contributions using eye-tracking within data visualization field include, among others, studies on graph readability~\cite{pohl2009comparing, huang2008beyond, huang2005people}, tree-drawing perception~\cite{burch2011evaluation, burch2013visual}, and visualization evaluation~\cite{kim2012does}.  The breadth and scope of eye tracking research is growing rapidly as eye-trackers become increasingly accurate, fast, and affordable.  

Eye-tracking data is generally analyzed using one of two approaches: point-based and area-of-interest (AOI)-based~\cite{blascheck2014state}. Point-based analyses consider gaze samples individually albeit AOI analyses accumulate gazes into areas of interests and then administer at this high abstraction level. 

Traditionally, analysts define AOIs manually over stimuli although automatic AOI defining is possible using gaze-clustering algorithms over available eye-tracking data~\cite{privitera2000algorithms, santella2004robust, drusch2014analysing}. Our work is adjoining to the methods of relating gazes automatically to the semantics of the underlying data that generate visual stimuli. Steichen et al.~\cite{steichen2013user}, and Kurzhals et al.'s work~\cite{kurzhals2014iseecube} implies that AOIs can be dynamically defined for such circumstances. Albeit, their work neither formalize any procedures nor compute feasibility. More precisely, for dynamic 3D stimuli, Stellmach et al. introduced the concept of object of interests (OOI) where gazes can be accumulated to 3D objects in the scene~\cite{stellmach20103d}. Moreover, Bernhard et al. proposed identical gaze-to-object mapping (GTOM) in the context of comprehending people's perception to objects in 3D environments~\cite{bernhard2014gaze}. 

Recently, we proposed feasibility of automatically relating gazes to semantic contents of a regular 2D information visualization using DOI-based analysis. We also described, the practicability of instrumenting source code of generating visualization to generate DOIs in run-time. We compared the automatic DOI-based approach with conventional manual approach for analyzing eye-tracking data and the results show that the automatic DOI-based analysis data were similar to manual analysis data.

The concept of DOI is similar to mapping gazes to semantic information from Game engines~\cite{sundstedt2013visual}. Albeit, typical information visualization does not have access to such game engines. Moreover, our work differs from gaze analysis of 3D games on the concept of instrumentation and dynamic analysis. 

Plenty of visualization and visual analytics tools exist for both point-based and AOI-based eye-tracking analysis. Blascheck et al. presented an exhaustive survey of such methods~\cite{blascheck2014state}. Since DOI is analogue to AOI, visualization methods for AOI analyses are most pertinent to our work. Examples of such methods include scanpaths, scarf-plots, AOI-rivers and AOI transition matrices. Moreover, visual analytics principles and systems for AOI data is also relevant to our work. Such analytics principles are proposed by Andrienko et al.~\cite{andrienko2012visual}, Weibel et al.~\cite{weibel2012let}, and Kurzhals et al.~\cite{kurzhals2014iseecube}. 

To analyze DOI data with existing visualization techniques, analysts are required to perform some visual tasks. Visual tasks may vary depending on the data type used to generate visualizations. Shneiderman described a high level task taxonomy over different types of data such as 1D, 2D, 3D, temporal, multi-dimensional, and network~\cite{shneiderman1996eyes}. Moreover, Brehmer et al. recently proposed multi-level typology that can be aided to create a complete task description~\cite{brehmer2013multi}. Amar et al. proposed ten low-level visual-analytics tasks types as questions experimenters may ask while performing experiments with tabular data~\cite{amar2005low}. Finally, task taxonomies exist for some specific categories of visualizations, such as graph visualization~\cite{lee2006task}, group-level graph~\cite{saket2014group}, and multidimensional data visualization~\cite{ward2002taxonomy}. However, task taxonomy exist neither for AOI data nor DOI data. In this paper, we proposed a task taxonomy specific to DOI data which will aid analysts to perform DOI-based visualization. Moreover, The focus of the work to establish DOI-based a feasible analysis method for eye-tracking data, and provide a formal definition, task taxonomies and visualization techniques for DOI-based analysis. 





\bibliographystyle{abbrv}
%%use following if all content of bibtex file should be shown
%\nocite{*}
\bibliography{bib}
\end{document}

