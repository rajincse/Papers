\chapter{Literature Review}
Eye-tracking technology is gradually getting more accurate, faster and cheaper \cite{Duch07}. As a result, it has been a popular tool in many research fields such as psychology, neuroscience, human-computer interaction (HCI), and data visualization \cite{Duch02}. Specifically, Duchowski categorized eye-tracking applications into two classes: interactive and diagnostic \cite{Duch02}.

Interactive eye-tracking has been used as an alternative to pointing devices (e.g. mouse, touch-interface) in 2D \cite{Jacob91} and 3D \cite{Bolt90, Tan00}, and even as a text-input device \cite{Maj02}. However, eye-tracking generally proved ineffective compared to traditional selective devices (e.g. mouse, touch, and keyboard) due to the difficulty to differentiate between view-gazes and interaction-gazes. This is known as the ``The Midas Touch Problem'' \cite{Jacob91}.

The proposed research in this dissertation primarily focuses on eye-tracking's diagnostic role. The diagnostic use of eye-tracking is often involved in eye-tracking user studies. The most common form of a diagnostic eye-tracking study is a user solving visual tasks by observing visual stimuli on a computer screen while an eye-tracker records the user�s gaze positions. Then, analyzers process the gaze data offline to understand how people observed the stimuli and solved the tasks \cite{Duch07}. In this way, eye-tracking was used to understand how people recognize faces \cite{Guo14, Sha14}, how attention changes with emotion \cite{Ver13}, how diseases may affect perception \cite{Kim14}, and how student learn from visual contents \cite{Zaw15, May10, vGo10, Con13}.   