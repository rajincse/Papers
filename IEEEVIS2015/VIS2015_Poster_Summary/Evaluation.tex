\section{Evaluation}
\label{sec:Evaluation}
We collected the data from 9 graduate and undergraduate students with age ranging from 20 years to 30 years. Six of them were males and three females. We used a light weight 60Hz EyeX Tobii eye-tracker.The first users were asked to solve four types of tasks using the PivotPaths diagram described in Section~\ref{sec:Methods}. Prior to the tasks, the users were given a short training with a shorter version of the original study. First, given three famous actors of 80's also known as ``Brat Pack'' they have to find the famous movies they have acted in the 80s (Brat Pack task). Second, given two movie names the users had to find five commonalities between them. They had to find two actors, two genres and one director (Commonality task). Third, given a director name and a list of three actors, the users had to find a ranking based on their collaboration with the director (Ranking task). Fourth, given three movie names the users had to recommend a fourth movie (Recommendation task). The user study session was approximately one hour long. Each of the user was awarded with \$10 for their time. 

We had total 11 subjects for the second user study similar to the age range of the first study. Three of them were males and eight females. We had one male and one female who also participated in the first user study. For the second user study, there were two categories: offline and real-time. Six subjects participated in the offline and other five subjects on the real-time. In second study, we gather data from first user study one instance per task and show to the user in random sequence using the time-line visualization described in Section~\ref{sec:Methods}. We also asked the user to think-out-loud during the study. We noted the answers, and recorded voice and the screen video of seconds users during the study . In the offline category, there were data of five subjects from the first study. The second users were able to forward or rewind the time in the visualization at any time. On the other hand, the second users were shown the data of one fewer subject data from the offline and the Brat Pack task data was not shown. Prior to the tasks, the second users were given a short description of the first user study and the task instances. They were asked to browse imdb.com for ten minutes before the training. All of the users were asked to perform four tasks. First, given a random sequence of task data they had to identify when the first users were working on a task and what the task it was for each first user. Second, if they identify a task data of Brat Pack they had to find which actors' names were mentioned in the first study and which three movies related with the actors were answered. Third, for Commonality task data they had to find the common elements. Fourth, for Ranking task the second users had to find the three actors who were mentioned in the first user study. Finally, for Recommendation task they had to identify the fourth movie recommended by each first users. The offline session lasted for one hour and real-time session for fifty minutes. All of the second users were awarded \$10 for their time. 

