\section{Related Work}

Empirical evaluations of visualization tools are always helpful to understand the potentials and limitations. One of the specific challenges in the evaluation of InfoVis is to improve user testing~\cite{plaisant2004challenge}. For improving the user testing we can argue to know how well the user performing. Costagiola et al. shows an approach of improving performance of students while monitoring their tests using data visualization~\cite{costagliola2009monitoring}. We use this idea in the domain of eye-tracking user studies. For the design of the user study we have temporal data. From the temporal data, users are supposed to have an idea about the tasks. Shneiderman described about the types of task in case of temporal data~\cite{shneiderman1996eyes}. Lee et al. also provides a task taxonomy in graph visualization~\cite{lee2006task}.  

