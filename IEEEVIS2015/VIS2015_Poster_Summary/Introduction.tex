\section{Introduction}

Evaluation is one of the constitutive part of InfoVis research. InfoVis researchers evaluate visualization techniques empirically with user studies consisting human subjects. An InfoVis user study consists of several visual tasks. To conduct an user study, the human subjects often requires training on the visualization tools and techniques. Even if proper trainings are provided to users, it is almost infeasible to train an user perfectly about an unknown visualization system. We propose an idea whether we can detect such deficiency of a particular user. To tackle this problem we need to overcome two challenges. First, we need information of which elements users are looking at. Second, we need information regarding relationship between the viewing patterns and the visual tasks of users. For the avoidance of the first challenge we chose a data-set of a user study conducted using eye-tracking device. In this paper we approach the second challenge by demonstrating the method of automated retrieval of information about the visual task from a user study data-set. 

An eye-tracking device or eye-tracker provides data regarding gaze positions of users. The role of eye-tracking as diagnostic tool provided a new edge of innovation to multiple disciplines such as psychology, cognitive science, human-computer interaction (HCI), and InfoVis research. Generally two categories of analyses is widely used: point-based analysis and area of interest (AOI) analysis~\cite{blascheckstate}. However both of the categories require significant amount of manual input from the experimenters and impossible to use in an real-time scenario. Alam et al.~\cite{alamdata} proposed a data of interest (DOI) analysis of eye-tracking data which provide the scope of analysis in real time. DOI-based analysis of eye-tracking data works on the data-space instead of image space. DOI-based analysis method can be instrumented in the visualization system. The method provides fuzzy scores of elements which viewers are probably put their interest. 

We demonstrate our method of automated information retrieval from eye-tracking user study data we conducted a meta user study over the user study data-set from Alam et al.~\cite{alamdata}. More in-depth description of the meta user study is provided in Section~\ref{sec:Evaluation}. 
