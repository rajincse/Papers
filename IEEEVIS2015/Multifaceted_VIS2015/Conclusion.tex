\section{Conclusion}
In visualizations that are open to instrumentation, gaze information provided by an eye-tracker can be used to automatically detect what visual objects users are likely to be viewing. Such detection can provide results that are almost as accurate as annotations created by human coders, provided that detection is done ``intelligently'', by using gazes together with a prediction of which objects are likely to be viewed at a given time. Such predictions can be made based on users' previous viewed objects and interactions. Data collected in this way is highly granular and has semantic content because it is linked to the data underlying the visualization. For this reason, and because it doesn't not require any human pre-processing, object viewing data can be collected and analyzed efficiently for many subjects, using interactive visualizations, for long analytic session, and could be used in studies that explore how analysts hypothesize about data using complex visual analytics systems.    